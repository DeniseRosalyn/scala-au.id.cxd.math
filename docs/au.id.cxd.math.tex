\documentclass[a4paper]{article}


\title{au.id.cxd.math}
\date{03/10/2014}
\author{Chris Davey}

\usepackage[pdftex]{graphicx}

\usepackage{times}

\usepackage{mathptmx}

\usepackage{amsmath}

\usepackage{enumitem}

\usepackage{graphicx}


\pagestyle{headings}
\begin{document}


\section{Methods of Counting}

\subsection{package au.id.cxd.math.count}

The count package contains a series of modules dedicated to methods of counting.

\subsubsection{Factorial}

The factorial operation is provided as $n!$ implementing:
$$
\prod_{i=1}^{n-1} (n-i)
$$
The Factorial implementation will memoize results, allowing for efficient reuse during runtime. 

\subsubsection{Choose}

The choose module implements $n \choose m$, how many ways can m items be selected with replacement from a set of n items. \\
Determined as:
$$
\frac{n!}{m!n!}
$$

\subsubsection{Permutation}

The method of selecting m ordered items from a set of n ordered items $P {n \choose m}$.\\
$$
\frac{n!}{(n-m)!}
$$



\section{Overview}

This library implements a number of functions for mathematical operations, it has dependencies upon the scala-nlp-breeze matrix library, and it's purpose is to explore the implementation of various mathematical functions.

This document is structured with the headings corresponding to the major package names in the project.

\section{Methods of Counting}

\subsection{package au.id.cxd.math.count}

The count package contains a series of modules dedicated to methods of counting.

\subsubsection{Factorial}

The factorial operation is provided as $n!$ implementing:
$$
\prod_{i=1}^{n-1} (n-i)
$$

\subsubsection{Choose}

The choose module implements $n \choose m$, how many ways can m items be selected with replacement from a set of n items. \\
Determined as:
$$
\frac{n!}{m!n!}
$$

\subsubsection{Permutation}

The method of selecting m ordered items from a set of n ordered items $P {n \choose m}$.\\
$$
\frac{n!}{(n-m)!}
$$

\section{Probability}

\subsection{package au.id.cxd.math.probability}

This package provides a series of modules that support operations for inference via probability, and for estimation of distributions.

\subsubsection{Inequalities}

The class TchebysheffInequality implements a simple estimation of a pdf using the inequality rule:\\
$$
P(\mu - k\sigma < Y < \mu + k\sigma) \ge 1 - \frac{1}{k^2}
$$
Which can be restated as:
$$
P(lower < Y < upper) \ge 1 - \frac{1}{k^2}
$$
The value of $k$ is derived from either upper and lower bounds since:
$$
k = \frac{upper - \mu} {\sigma}
$$
After determining the value of $k$ the probability can be estimated by substituting
$$
p = 1 - \frac{1}{k^2}
$$

\subsubsection{Discrete Distributions}

The discrete distributions packages contains the following. 

\subsubsection{Binomial}

The binomial module has the parameters $p$ for the prior proportion of successes and $n$ for the total number of trials and calculates the probability of $y$ successes 
$$
P(y; n; p) = \sum_{i=1}^n {n \choose y_i} p^y_i (1-p)^{n-y_i} 
$$


\subsubsection{Continuous Distributions}

\subsubsection{Inference}

\subsubsection{Regression}


\end{document}
